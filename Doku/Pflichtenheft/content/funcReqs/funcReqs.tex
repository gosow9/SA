\section{Produktanforderungen}
\label{sec:funcReqs}
Das Empfängermodul ist autark und kann auf einem Display Stundenpläne anzeigen.
Diese werden über eine drahtlose, bidirektionale Schnittstelle gesendet.
Der Sender wird mit einem Computer bedient.
Besteht das System aus mehreren Empfängern, so kann das Sendemodul diese unabhängig voneinander selektieren.

\subsection{Hardware}
\textbf{Sender}
	\begin{itemize}
		\item Schnittelle zum Computer
		\item Sendemodul
	\end{itemize}

\textbf{Empfänger}
	\begin{itemize}
		\item Mikrocontroller oder Vergleichbares (Prozessor, Speicher, usw.)
		\item E-Paper-Display
		\item Energy-Harvesting-Einheit
		\item Energiespeicher
		\item Empfangsmodul
	\end{itemize}

\subsection{Software}
\textbf{Sender}
\begin{itemize}
	\item Treiber für Sendemodul
\end{itemize}

\textbf{Empfänger}
\begin{itemize}
	\item Firmware für Mikrocontroller
\end{itemize}

\subsection{Varianten/Optionen}
Ist der Prototyp funktionsfähig, soll zu einem späteren Zeitpunkt auch möglich sein, verschiedene Bildschirmgrössen zu verwenden, wobei sich auch Anzeige nicht nur auf Raumbelegungspläne beschränkt.
Deshalb soll das System und insbesondere die Software so flexibel wie möglich entwickelt werden.

\subsection{Dokumentation}
Die Dokumentation beinhaltet sämtliche Überlegungen, Abklärungen, Berechnungen und Untersuchungen, welche im Laufe der Semesterarbeit gemacht wurden.