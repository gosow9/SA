%%%%%%%%%%%%%%%%%%%%%%%%%%%%%%%%%%%%%%%%%%
% Dokument
%%%%%%%%%%%%%%%%%%%%%%%%%%%%%%%%%%%%%%%%%%
% Geometrie
\newcommand{\paperFormat}{a4paper}
\newcommand{\lPageMargin}{20mm}
\newcommand{\rPageMargin}{20mm}
\newcommand{\tPageMargin}{20mm}
\newcommand{\bPageMargin}{20mm}

\documentclass[11pt,oneside]{scrartcl}

\newcommand{\newpar}{\par\par}

\usepackage[pdftitle={\titleinfo},%
			pdfauthor={\authorinfo},%
			pdfcreator={pdfLatex, LaTeX with hyperref},
			pdfsubject={\subjectinfo},
			plainpages=false,
			pdfpagelabels,
			colorlinks,
			linkcolor=black,
			filecolor=black,
			citecolor=black,
			urlcolor=black]{hyperref}
						

% Headings
\usepackage{scrlayer-scrpage}


%%%%%%%%%%%%%%%%%%%%%%%%%%%%%%%%%%%%%%%%%%
% Package's
%%%%%%%%%%%%%%%%%%%%%%%%%%%%%%%%%%%%%%%%%%
\usepackage{ucs}
\usepackage[utf8x]{inputenc}
\usepackage[T1]{fontenc}

\usepackage[free-standing-units=true, use-xspace=true]{siunitx}

\usepackage{layout}
\setlength{\parindent}{0em}

\renewcommand{\baselinestretch}{1.2}
\renewcommand{\arraystretch}{1}

%Damit \today ein Deutsch Formatiertes Datum zurueckgibt.
\usepackage[ngerman, num, orig]{isodate}
\usepackage[german, ngerman]{babel}
\monthyearsepgerman{\,}{\,}

\usepackage{amssymb,amsmath,fancybox,graphicx,wrapfig,color,lastpage,verbatim,epstopdf,a4wide,tabularx}
\usepackage{wasysym} %Checkboxen
\usepackage[usenames,dvipsnames]{pstricks}
\usepackage{setspace}
\usepackage{epsfig}
\usepackage{pst-pdf}
\usepackage{pst-all}
\usepackage{pstricks-add}
\usepackage{supertabular}
\usepackage[font=small,labelfont=bf]{caption}
\usepackage[font=footnotesize]{subfig}
\usepackage{footnote}
\usepackage{float}
\usepackage{multirow}
\usepackage{pdfpages}
\usepackage{pgf,tikz}
\usepackage{color}
\usepackage{titletoc}

\usepackage[makeroom]{cancel}
\usepackage{array}
\usepackage{trfsigns}
\usepackage{textcomp}

%Querformat
\usepackage{pdflscape}

\renewcommand{\captionfont}{\scriptsize\slshape}
	
\setlength{\unitlength}{1mm}

%Inhaltsverzeichnis
\setcounter{secnumdepth}{4}
\setcounter{tocdepth}{2}

%Geometrie
\usepackage[\paperFormat,left=\lPageMargin,right=\rPageMargin,top=\tPageMargin,bottom=\bPageMargin,includeheadfoot]{geometry}


%%%%%%%%%%%%%%%%%%%%%%%%%%%%%%%%%%%%%%%%%%%%%%%%%%%%%%%%%%%%%%%%
% Environment Numbering
%%%%%%%%%%%%%%%%%%%%%%%%%%%%%%%%%%%%%%%%%%%%%%%%%%%%%%%%%%%%%%%%

%Abbildungsnumerierung anhand Kapitel
\renewcommand{\thefigure}{\arabic{section}.\arabic{figure}}
\makeatletter \@addtoreset{figure}{section} \makeatother

%Gleichungen anhand Kapitel
\AtBeginDocument{\numberwithin{equation}{section}}
\AtBeginDocument{\numberwithin{figure}{section}}
\AtBeginDocument{\numberwithin{table}{section}}


%%%%%%%%%%%%%%%%%%%%%%%%%%%%%%%%%%%%%%%%%%%%%%%%%%%%%%%%%%%%%%%%
% Farben
%%%%%%%%%%%%%%%%%%%%%%%%%%%%%%%%%%%%%%%%%%%%%%%%%%%%%%%%%%%%%%%%
\definecolor{black}{rgb}{0,0,0}
\definecolor{red}{rgb}{1,0,0}
\definecolor{white}{rgb}{1,1,1}
\definecolor{grey}{rgb}{0.8,0.8,0.8}


%%%%%%%%%%%%%%%%%%%%%%%%%%%%%%%%%%%%%%%%%%%%%%%%%%%%%%%%%%%%%%%%
% Einheiten
%%%%%%%%%%%%%%%%%%%%%%%%%%%%%%%%%%%%%%%%%%%%%%%%%%%%%%%%%%%%%%%%


%Spannung
\DeclareMathOperator{\V}{\volt}
\DeclareMathOperator{\mV}{\milli \volt}
\DeclareMathOperator{\uV}{\micro \volt}

%Strom
\DeclareMathOperator{\A}{\ampere}
\DeclareMathOperator{\mA}{\milli \ampere}
\DeclareMathOperator{\uA}{\micro \ampere}
\DeclareMathOperator{\nA}{\nano \ampere}

%Zeit
\DeclareMathOperator{\s}{\second}
\DeclareMathOperator{\ms}{\milli \second}
\DeclareMathOperator{\us}{\micro \second}
\DeclareMathOperator{\ns}{\nano \second}

%Kapazitaet
\DeclareMathOperator{\mF}{\milli \farad}
\DeclareMathOperator{\uF}{\micro \farad}
\DeclareMathOperator{\nF}{\nano \farad}
\DeclareMathOperator{\pF}{\pico \farad}
\DeclareMathOperator{\fF}{\femto \farad}

%Induktivitaet
\DeclareMathOperator{\mH}{\milli \henry}
\DeclareMathOperator{\uH}{\micro \henry}
\DeclareMathOperator{\nH}{\nano \henry}

%Widerstand
\DeclareMathOperator{\MO}{\mega \ohm}
\DeclareMathOperator{\kO}{\kilo \ohm}
\DeclareMathOperator{\mO}{\milli \ohm}
\DeclareMathOperator{\Ohm}{\ohm}
%Strecke
\DeclareMathOperator{\km}{\kilo \meter}
\DeclareMathOperator{\cm}{\centi \meter}
\DeclareMathOperator{\mm}{\milli \meter}

%Frequenz
\DeclareMathOperator{\GHz}{\giga \hertz}
\DeclareMathOperator{\MHz}{\mega \hertz}
\DeclareMathOperator{\Hz}{\hertz}
\DeclareMathOperator{\kHz}{\kilo \hertz}
\DeclareMathOperator{\mHz}{\milli \hertz}

%Leistung
\DeclareMathOperator{\kW}{\kilo \watt}
\DeclareMathOperator{\mW}{\milli \watt}
\DeclareMathOperator{\uW}{\micro \watt}
\DeclareMathOperator{\W}{\watt}

%Kreisfrequenz
\DeclareMathOperator{\rpers}{\radianpersecond}

%DeziBel
\DeclareMathOperator{\dB}{\deci \bel}
\DeclareMathOperator{\dBm}{\deci \bel \milli}

%Bit
\DeclareMathOperator{\Bit}{\text{Bit}}
\DeclareMathOperator{\kBit}{\text{kBit}}
\DeclareMathOperator{\MBit}{\text{MBit}}
\DeclareMathOperator{\Byte}{\text{Byte}}
\DeclareMathOperator{\kByte}{\text{kByte}}
\DeclareMathOperator{\MByte}{\text{MByte}}
\DeclareMathOperator{\ppm}{\text{ppm}}