\chapter*{Abstract}

\section*{Introduction}
In company buildings, universities and other similar facilities, it is usual to attach occupancy schedules next to room entrances.
These schedules are mostly printed on paper, or in more modern buildings, displayed on a screen.
Using paper, it is impracticable to take short-term changes into account, since every adjustment needs to be made by printing a new schedule.
Screens on the other hand need either to be connected to a power supply, which entails extensive installation work, or are powered by a battery, which needs to be replaced from time to time.
All these disadvantages could be avoided by using a screen which harvests its energy from the indoor light, and is updated via a wireless interface.
The screen should therefore be completely wireless, which simplifies the installation.

\section*{Approach}
Solar cells should harvest the required energy which can be store in a super-capacitor.
Because this harvesting unit cannot provide unlimited energy, the whole system with display, microcontroller and wireless interface should consume as little energy as possible.
To achieve this, a e-paper display is used, which only needs energy to update the screen, but not to maintain the displayed data.
By using a wake up receiver, it is now possible to disconnect the energy storage from the system, since it can be reconnected again when needed if a wake up interrupt is detected.
In a nutshell, the display, microcontroller and interface can be completely disconnected from the power supply when not used.
In this state, only the leakage of the super-capacitor and the wakeup receiver in listening mode consume power.
This consumption is in the micro watt range.

\section*{Conclusion}
