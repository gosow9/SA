In the course of this thesis, a working prototype of a occupancy schedule was developed.
As components served several development kits and simple prints.
Using a low power wakeup receiver turned out to be the right approach, since the power consumption during standby mode could that way be reduced to a minimum of $4.5\,\mu\text{W}$.
This power loss is easily compensated by the solar cells which deliver $485.52\,\mu\text{W}$ given an illuminance of 200\,lux.

With a self-holding circuit, it was possible to turn off parts of the receiver completely and switch them back on with the a wakeup interrupt.
