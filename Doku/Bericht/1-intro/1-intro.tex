\chapter{Introduction}
In company buildings, universities and other similar facilities, it is usual to attach occupancy schedules next to room entrances.
These schedules are mostly printed on paper, or in more modern buildings, displayed on a screen.
Using paper, it is impracticable to take short-term changes into account, since every adjustment needs to be made by printing a new schedule.
Screens on the other hand need either to be connected to a power supply, which entails extensive installation work, or are powered by a battery, which needs to be replaced from time to time.
All these disadvantages could be avoided by using a screen which can be updated wireless and is in itself autarkic.
The goal of this semester thesis is to develop a functional prototype of such a schedule.

\section{Task analysis}
The prototype should combine both the advantages of the paper method and the screen.
Updating should happen via a wireless interface while a expensive installation should be avoided.
The screen should therefore be placed on the desired location  just like a paper schedule.
To consume as low power as possible, the microcontroller, display and interface should be switched off when not needed.

\section{Approach}

