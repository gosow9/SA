\chapter{Introduction}
In company buildings, universities and other similar facilities, it is usual to attach occupancy schedules next to room entrances.
These schedules are mostly printed on paper, or in more modern buildings, displayed on a screen.
Using paper, it is impracticable to take short-term changes into account, since every adjustment needs to be made by printing a new schedule.
Screens on the other hand need either to be connected to a power supply, which entails extensive installation work, or are powered by a battery, which needs to be replaced from time to time.
All these disadvantages could be avoided by using a screen which can be updated wireless and is in itself autarkic.
The goal of this semester thesis is to develop a functional prototype of such a schedule.
To keep the energy consumption at a minimum, a low-power wakeup receiver is used.

\section{Task analysis}
The prototype should combine both the advantages of the paper method and the screen.
Updating should happen via a wireless interface while a expensive installation should be avoided.
The screen should therefore be placed on the desired location  just like a paper schedule.
To consume as low power as possible, the microcontroller, display and interface should be switched off when not needed.

\section{Approach}
To simplify the description, from now on the display with all its components is called the receiver, while the other end, a computer with additional hardware is called the transmitter.
This does not mean, that the communication is uni-directional, since both ends should be able receive and transmit data, but more that the general data transfer happens from the computer to the screen.

\paragraph{Transmitter}
This end only consists of a computer, a wireless interface for the main communication and the transmitter module of the wakeup receiver.
A short python script enables the user to define which data the receiver should print over a fixed template.
Information is transferred to a wireless interface which transmits it to the receiver end after the receiver is woken up.

\paragraph{Receiver}
This part of the prototype should consume as low power as possible.
An e-paper display is used, because it is bi-directional, which means it can maintain its image even when disconnected of the power supply.
A wireless module receives information from the transmitter and, if needed, sends back feedback about the status.
This connection should therefore be bi-directional.
To process the incoming data and display it correctly on the screen, a microcontroller is needed.
All these components are completely disconnected from the power supply when not used.
The power is provided by a super capacitor which is charged by solar cells.
When switched off, the only component which actively consumes power is a wakeup receiver.
This module is constantly in listening mode and waits for a wakeup event.
Is such an event received, it activates a self-holding circuit which connects the power supply to the other components.
Data can now be exchanged over a different channel, established by the wireless module.
After the data is processed and the display is refreshed, the microcontroller deactivates the self-holding circuit and disconnects in this step itself and all other components except the wakeup receiver from the power source. 