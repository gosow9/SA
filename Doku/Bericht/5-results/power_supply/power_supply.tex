There is a voltage drop over the ADP5090 ($V_{\text{BAT}}-V_{\text{SYS}}$ in Figure \ref{development:discharge}) if a load is attached to the main output.
This drop causes the chip to switch off the output way too early.
Out of this reason, it was necessary to bypass the ADP5090 and connect the receiver circuit directly to the capacitor.
This way, the capacitor is not protected from overdischarging, but charging of the capacitor is still controlled by the ADP5090.

Figure \ref{results:v} shows a plot of the voltages of the power supply during one write cycle. 
\begin{figure}[ht]
	\centering
	\includegraphics[width=0.9\textwidth]{5-results/power_supply/plot/v.pdf}
	\caption{Power consumption during one write cycle.\label{results:v}}
\end{figure}
The voltage over the capacitor is labelled as $V_{\text{BAT}}$.
Additionally, the voltages after the step-up converter ($V_{\text{BOOT}}$) and LDO ($V_{\text{LDO}}$) were tracked.
The whole power supply manages to provide a stable voltage over the whole write cycle, despite high current peaks.