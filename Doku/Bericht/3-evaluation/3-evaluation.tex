\chapter{Evaluation}
This chapter lists the pros and cons of available Technologies.
On this basis was evaluated, which hardware is suitable.

\section{Wakeup Receiver}
In this semester thesis, two different implementations of the wakeup receiver technology were compared.
The AS3933 from AMS and the RFicient from the \acf{fraun} which kindly provided an evaluation kit before the actual release of the product.
This section first introduces both products, before comparing them with a series of tests in a for this application realistic environment.

\subsection{AS3933}
The AS3933 is a low frequency wakeup receiver, which uses \acs{ask} to modulate a carrier frequency between 15-150\,kHz.
The transmitter sends a manchester encoded, programmable wakeup pattern of length 16 or 32 bit.
If this pattern is detected on the receiver end, a wakeup interrupt is generated.
More important features on receiver end are:
\begin{itemize}
	\item[-] Receiver sensitivity $80\,\mu\text{VRMS}$ (equivalent to $-68.93\,\text{dBm}$)
	\item[-] Current consumption in 3-channel listening mode $2.3\,\mu\text{A}$
	\item[-] Operating supply range $2.4\,\text{V}-3.6\,\text{V}$
	\item[-] Three antennas (enables 3D detection)
	\item[-] Channels  individually selective to reduce power consumption
\end{itemize}
The low power consumption makes it possible run the receiver in listening mode below $8.3\,\mu\text{W}$\cite{as3933}.

The demo kit comes with a \acs{gui}, which enables the user to set the parameters as desired.
First trials showed, that only one of the  three antennas seemed to work.
According to the manual, it should have been possible to activate the other two channels as well, but the in the demo kit included \acs{gui} did not provide this option.
Consulting the technical support of AMS was unfortunately of no use.
Therefore all test were made with only one channel as good as possible.

\subsection{RFicient}
The Rficient from the \acs{fraun} uses \acs{ook} to modulate a 868\,MHz signal.
It can either run in pure wakeup mode, where the receiver generates an interrupt as soon as a code is received or a selective mode, where a 16 bit wakeup preamble needs to match the receiver. 
After the preamble is detected, the data rate can be changed to transfer more bits which can be sent over an \acs{spi}-bus to a connected device.
Data rates can be set in a range between $256\,\text{bp/s}-32\,\text{kbp/s}$.
the most important features are:
\begin{itemize}
	\item[-] Receiver sensitivity -80\,dBm
	\item[-] Energy consumption $3\,\mu\text{A}$ at $1.5\,\text{V}$ (data rate 1 kbit/s)
	\item[-] Unidirectional data transfer possible	
\end{itemize}
The power consumption therefore is in listening mode (data rate = 1 kb/s) $4.5\,\mu\text{W}$.

Just as the AS3933, the RFicient demo kit comes with a \acs{gui}, which enables the user to set the important parameters and even access the register directly \cite{rficient}.

\subsection{Test run in realistic environments}