\chapter{Evaluation}

\section{Wakeup Receiver}
In this semester thesis, two different implementations of the wakeup receiver technology were compared.
The AS3933 from AMS and the RFicient from the \acf{fraun} which kindly provided an evaluation kit before the actual release of the product.

\subsection{AS3933}
The AS3933 is a low frequency wakeup receiver, which uses \acs{ask} to modulate a carrier frequency between 15-150\,kHz.
The transmitter sends a manchester encoded, programmable wakeup pattern of length 16 or 32 bit.
If this pattern is detected on the receiver end, a wakeup interrupt is generated.
It is also possible to disable the the pattern decoder to run the chip in a frequency detection mode, where a wakeup interrupt is generated as soon as the specified frequency is received.
More important features on the receiver end are:
\begin{itemize}
	\item[-] Receiver sensitivity $80\,\mu\text{V$_{\text{RMS}}$}$
	\item[-] Current consumption in 3-channel listening mode $2.3\,\mu\text{A}$
	\item[-] Operating supply range $2.4\,\text{V}-3.6\,\text{V}$
	\item[-] Three antennas (enables 3D detection)
	\item[-] Channels  individually selective to reduce power consumption
\end{itemize}
The low power consumption makes it possible run the receiver in listening mode below $8.3\,\mu\text{W}$\cite{as3933}.

The demo kit comes with a \acs{gui}, which enables the user to set the parameters as desired and address the registers directly.
The range of the receiver of about 6\,m as first measurement turns out to be very limited.
Even with a 3db sensitivity boost on the receiver side, it is only possible to detect wakeup events from a distance of 9\,m.
The environment (indoor, outdoor) seems to make no huge difference.

As a result, the limited range makes the AS3933 unusable for the prototype. 

\subsection{RFicient}
The Rficient from the \acs{fraun} uses \acs{ook} to modulate a 868\,MHz signal.
It can either run in pure wakeup mode, where the receiver generates an interrupt as soon as a code is received or a selective mode, where a 16 bit wakeup preamble needs to match the receiver. 
After the preamble is detected, the data rate can be changed to transfer more bits which can be sent over an \acs{spi}-bus to a connected device.
This way, it is possible to transmit data bits after the actual wakeup.
Data rates can be set in a range between $256\,\text{bp/s}-32\,\text{kbp/s}$.
The most important features are:
\begin{itemize}
	\item[-] Receiver sensitivity -80\,dBm
	\item[-] Energy consumption $3\,\mu\text{A}$ at $1.5\,\text{V}$ (data rate 1 kbit/s)
	\item[-] Unidirectional data transfer possible	
\end{itemize}
The power consumption therefore is in listening mode (data rate = 1 kb/s) $4.5\,\mu\text{W}$.

Just as the AS3933, the RFicient demo kit comes with a \acs{gui}, which enables the user to set the important parameters and access the register \cite{rficient}.
First measurements showed that the range of the RFicient is far higher than the range of the AS3933.
It is therefore used in the prototype.

\section{Microcontroller}
To handle incoming data and write it to the display driver, some kind of microcontroller is needed.
A derivate of the MSP430 and one of the STM32 were looked at more closely.

\subsection{MSP430}
The MSP430FR6989 was up for selection because it is specifically developed for ultra-low-power applications.
Of particular interest is mainly the non volatile 128\,KB \acf{fram}.
Key-features are:
\begin{itemize}
	\item Approximately  $100\,\mu\text{A}/\text{MHz}$ 
\end{itemize}


\subsection{STM32}