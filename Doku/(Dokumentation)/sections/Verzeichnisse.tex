\section{Verzeichnisse}
%Abkürzungsverzeichniss, für schön ausrichten längste Abkürzung in eckige Klammer
\subsection{Abkürzungen}
\vspace{0.1cm}
\begin{acronym}[BA]
%\acro{Abkürzung}{Ausdruck ausgeschrieben}	
\acro{HSR}{Hochschule für Technik Rapperswil}
\end{acronym}	
\clearpage
\pagebreak


%Gleichungsverzeichnis
\subsection{\listequationsname}
\renewcommand*\listequationsname{}
\listofmyequations\thispagestyle{fancy}
\clearpage
\pagebreak

%Abbildungsverzeichnis
\subsection{\listfigurename}
\renewcommand*\listfigurename{}
\listoffigures\thispagestyle{fancy}
\clearpage
\pagebreak

%Tabellenverzeichnis
\subsection{\listtablename}
\renewcommand*\listtablename{}
\listoftables\thispagestyle{fancy}
\clearpage
\pagebreak

%Quellenverzeichnis
\subsection{Quellenverzeichnis}
\printbibliography[heading=subbibliography,keyword=lit, title={Literaturquellen}]
\printbibliography[heading=subbibliography,keyword=manual, title={Datenblätter}]
\printbibliography[heading=subbibliography,keyword=online, title={Onlinequellen}]
\printbibliography[heading=subbibliography,keyword=abb, title={Bildquellen}]

\clearpage